\documentclass[11pt, oneside]{article}   	% use "amsart" instead of "article" for AMSLaTeX format
\usepackage{geometry}                		% See geometry.pdf to learn the layout options. There are lots.
\geometry{letterpaper}                   		% ... or a4paper or a5paper or ... 
%\geometry{landscape}                		% Activate for rotated page geometry
%\usepackage[parfill]{parskip}    		% Activate to begin paragraphs with an empty line rather than an indent
\usepackage{graphicx}				% Use pdf, png, jpg, or eps§ with pdflatex; use eps in DVI mode
								% TeX will automatically convert eps --> pdf in pdflatex		
\usepackage{amssymb}

%SetFonts

%SetFonts


\title{SitePassword Modulo Bias}
\author{Alan H. Karp}
%\date{}							% Activate to display a given date or no date

\begin{document}
\maketitle
\abstract

SitePassword hashes  a super password, a per site nickname, and a per site userid to produce a site password.  The bytes produced by the hash need to be converted to a string of characters selected from the alphabet supported by the web site associated with that user account.  In some situations, {\em modulo bias} reduces the entropy of the site password.  This note shows that the effect is small enough to ignore for the range of values SitePassword supports.

\section{Introduction}

The original SitePassword algorithm used each byte of the hash produced by PBKDF2\footnote{Password Based Key Derivation Function} to compute the $i$-th character of the site password using $c_i = A[b_i  \bmod  |A|)]$, where $A$ is an array of characters representing the alphabet, $b_i$ is the byte value, and $|A|$ is the size of the alphabet.  If the size of the alphabet evenly divides $2^8 = 256$, every character in the alphabet has the same probability of being selected.  If it doesn't, then some characters have a higher probability than others.  This {\em modulo bias} reduces the entropy of the resulting string.  This document shows that modulo bias does not have a substantial effect on the guessability of the site password it generates.

\section{Recommended Approaches}

The recommended approach to eliminating modulo bias is {\em rejection sampling}.  Simply ignore any byte value index outside the range of the alphabet.  In other words, don't use any $b_i > |A|$.  Theoretically, this approach could require an unbounded number of bytes, but the number of bytes needed is quite modest in practice.  The recommended approach to mitigating modulo bias is to use a larger range of random numbers, say two bytes instead of one.  There is still bias, but the effect on guessability is substantially smaller.

Unfortunately, while rejection sample makes it harder to guess the site password, it makes it easier to guess the super password.  PBKDF2 has a runtime that depends on both the key size and iteration count, but the adversary only needs to compute the number of bytes actually used to construct the site password.  In the most likely case, fewer than half the bytes will be rejected when selecting a single character, but more than 10 times as many might be needed.  The additional bytes can be computed conditionally, but that step reduces the number of iterations that SitePassword can do and still meet the latency requirement.  Hence, it's important to use all the bytes of the hash in constructing the site password.

\section{Quantifying Modulo Bias}

There will be no need to use extra bytes to construct the site password if modulo bias doesn't weaken the site password very much.  One way to quantify that weakening is by looking at the entropy of the site password both with and without modulo bias.  Without bias, the entropy of an $L$-character site password is

\begin{equation}
E_0 = L \log_2 |A|
\end{equation}

The easiest way to understand modulo bias is to repeat the alphabet until the result is has more elements than the largest possible random value $R$ and truncate to size $R$.  Some numbers may appear one more time than others.  In those cases, the number of times a character appears is either
\begin{equation}
n_1 = \lfloor R/|A| \rfloor ~~\textrm{or}~~ n_2 = \lceil R/|A| \rceil,
\end{equation}
so that $n_1 = n_2$ if $|A|$ divides $R$ or $n_2 = n_1 + 1$ otherwise.  The probabilities that one of those characters is chosen is, 
\begin{equation}
p_1 = n_1/R ~~\textrm{or}~~ p_2 = n_2/R.
\end{equation}

The entropy of selecting a single character from each subset is then
\begin{equation}
E_1 = -(|A| - R \bmod |A]) p_1 \log_2 p_1 ~~~\textrm{or}~~ E_2 = - (R \bmod |A|) p_2 \log_2 p_2.
\end{equation}

Here $R \bmod|A|$ is the number of characters that appear an extra time.  Note that if $|A|$ divides $R$, $E_2 = 0$ and $E_1 = E_0$.  The entropy of an $L$-character site password with modulo bias is $E_b = L(E_1 + E_2)$.  

\begin{figure}
    \centering
    \includegraphics[width=0.5\textwidth]{moduloBias.png} 
    \caption{Modulo bias for SitePassword for an $L$-character site password.  The green line is the entropy loss; the purple line the reduction in the number of guesses.  Only values of $10 \leq |A| \leq 96$ are relevant to SitePassword.}
    \label{fig}
\end{figure}

We can now plug in some numbers.  SitePassword uses $R = 256$ and allows alphabets in the range $10 \leq |A| \leq 96$.  With the default values of $|A| = 62$ and $L = 12$, $E_0 - E_b = .05$ which requires 96\% of the $2^{71}$ guesses needed if there was no modulo bias.  The worst case in the allowed range is at $|A| = 96$ where $E_0 - E_b = 0.28$, which translates into 82\% as many guesses.  The greater loss in guesses is offset by the larger alphabet size leaving the entropy with modulo bias greater than 78 bits.  Figure~\ref{fig} shows the entropy loss, $E_0 - E_b$, in green and the ratio of the number of guesses needed, $2^{-(E_0-E_b)}$, in purple for $R = 256$ and $L = 12$.

\section{Conclusions}

There are cases where a small modulo bias can be a serious problem, such as in choosing a nonce for a stream cypher.  That is not the case for SitePassword.  Even the most extreme modulo bias reduces the entropy of the site password by less than one bit.  Even so, the SitePassword algorithm was changed to use 2 bytes per character, which makes the modulo bias considerably smaller.

\end{document}  